% Options for packages loaded elsewhere
\PassOptionsToPackage{unicode}{hyperref}
\PassOptionsToPackage{hyphens}{url}
%
\documentclass[
  11pt,
  brazil,
  a4paper,
]{article}
\usepackage{lmodern}
\usepackage{amssymb,amsmath}
\usepackage{ifxetex,ifluatex}
\ifnum 0\ifxetex 1\fi\ifluatex 1\fi=0 % if pdftex
  \usepackage[T1]{fontenc}
  \usepackage[utf8]{inputenc}
  \usepackage{textcomp} % provide euro and other symbols
\else % if luatex or xetex
  \usepackage{unicode-math}
  \defaultfontfeatures{Scale=MatchLowercase}
  \defaultfontfeatures[\rmfamily]{Ligatures=TeX,Scale=1}
  \setmainfont[Numbers=Proportional,BoldFont = UnBPro-Bold,ItalicFont =
UnBPro-RegularItalic,BoldItalicFont = UnBPro-BoldItalic]{UnB Pro}
  \setsansfont[]{UnB Pro-Light}
\fi
% Use upquote if available, for straight quotes in verbatim environments
\IfFileExists{upquote.sty}{\usepackage{upquote}}{}
\IfFileExists{microtype.sty}{% use microtype if available
  \usepackage[]{microtype}
  \UseMicrotypeSet[protrusion]{basicmath} % disable protrusion for tt fonts
}{}
\usepackage{xcolor}
\IfFileExists{xurl.sty}{\usepackage{xurl}}{} % add URL line breaks if available
\IfFileExists{bookmark.sty}{\usepackage{bookmark}}{\usepackage{hyperref}}
\hypersetup{
  pdftitle={TAU 0006 História da Arquitetura e da Arte II},
  pdfauthor={Fau--UnB},
  pdflang={pt-BR},
  hidelinks,
  pdfcreator={LaTeX via pandoc}}
\urlstyle{same} % disable monospaced font for URLs
\usepackage[top=80pt,left=80pt,textwidth=440pt,textheight=680pt]{geometry}
\usepackage{longtable,booktabs}
% Correct order of tables after \paragraph or \subparagraph
\usepackage{etoolbox}
\makeatletter
\patchcmd\longtable{\par}{\if@noskipsec\mbox{}\fi\par}{}{}
\makeatother
% Allow footnotes in longtable head/foot
\IfFileExists{footnotehyper.sty}{\usepackage{footnotehyper}}{\usepackage{footnote}}
\makesavenoteenv{longtable}
\usepackage[normalem]{ulem}
% Avoid problems with \sout in headers with hyperref
\pdfstringdefDisableCommands{\renewcommand{\sout}{}}
\setlength{\emergencystretch}{3em} % prevent overfull lines
\providecommand{\tightlist}{%
  \setlength{\itemsep}{0pt}\setlength{\parskip}{0pt}}
\setcounter{secnumdepth}{2}
\fancyhf{}
\lhead{\XeTeXglyph \XeTeXglyphindex "ass_completa_CONT" \relax}
\lfoot{\small{FAU/THAU/PP/2020}}
\rfoot{\small{\thepage}}
\renewcommand{\headrulewidth}{0.5pt}
\renewcommand{\footrulewidth}{0.5pt}
\newcommand{\pcdoc}{Pandoc-crossref documentation}
\ifxetex
  % Load polyglossia as late as possible: uses bidi with RTL langages (e.g. Hebrew, Arabic)
  \usepackage{polyglossia}
  \setmainlanguage[]{brazil}
  \setotherlanguage[]{english}
  \setotherlanguage[]{french}
  \setotherlanguage[]{german}
  \setotherlanguage[]{italian}
  \setotherlanguage[]{latin}
  \setotherlanguage[]{portuguese}
  \setotherlanguage[]{spanish}
\else
  \usepackage[shorthands=off,main=brazil]{babel}
\fi
\usepackage[style=chicago-notes,strict,backend=biber,babel=other,bibencoding=inputenc]{biblatex}
\addbibresource{leituras.bib}

\title{TAU 0006 História da Arquitetura e da Arte II}
\usepackage{etoolbox}
\makeatletter
\providecommand{\subtitle}[1]{% add subtitle to \maketitle
  \apptocmd{\@title}{\par {\large #1 \par}}{}{}
}
\makeatother
\subtitle{Plano de ensino}
\author{Fau--UnB}
\date{1.º/2020 · Turma C · 3.ª/5.ª 20h50--22h30}

\begin{document}
\maketitle
\begin{abstract}
História das cidades, dos edifícios e da edificação na tradição europeia
a partir do início do Renascimento na Itália até o século XVIII.
Transposição da tradição técnico-construtiva e arquitetônica da Europa
para as colônias americanas de fala inglesa e francesa.
\end{abstract}

\hypertarget{apresentauxe7uxe3o}{%
\section{Apresentação}\label{apresentauxe7uxe3o}}

Esta disciplina dá sequência ao conteúdo de história da arquitetura em
ordem cronológica e visando a atingir uma perspectiva mundial. O
fundamento da cadeia de disciplinas de Teoria e História da Arquitetura
e do Urbanismo, incluindo Estética e História das Artes, é subsidiar,
por meio do conhecimento de repertórios formais e teorias, a qualidade
da concepção e da prática de arquitetura, urbanismo e paisagismo, bem
como fomentar a reflexão crítica e a pesquisa. Visamos a constituir uma
cultura histórica geral, além de repertórios de soluções projetuais e,
mais importante, fomentar a reflexão crítica sobre a produção do
ambiente construído ao longo do tempo.

Em TAU 0006, a formação do espaço urbano moderno e a evolução do ofício
da arquitetura são os temas dominantes que norteiam o conteúdo. Embora a
relação entre o contexto político ou socioeconômico e a produção das
edificações e das cidades seja uma abordagem presente ao longo do
semestre, vamos nos afastar da equivocada percepção de uma pretensa
correspondência direta entre estilos arquitetônicos e ideologias ou
sistema político-econômicos. Por fim, vamos ultrapassar os limites da
ementa, procurando manter sempre que possível uma perspectiva mundial.

\hypertarget{equipe}{%
\subsection{Equipe}\label{equipe}}

\hypertarget{professor}{%
\paragraph{Professor}\label{professor}}

Pedro P. Palazzo

\hypertarget{estagiuxe1rio-docente}{%
\paragraph{Estagiário docente}\label{estagiuxe1rio-docente}}

Sylvio Farias

\hypertarget{monitoras}{%
\paragraph{Monitoras}\label{monitoras}}

Nathália Bonfim e Gabriella Parpinelli

\hypertarget{objetivos-de-aprendizagem}{%
\subsection{Objetivos de aprendizagem}\label{objetivos-de-aprendizagem}}

O objetivo geral da disciplina é adquirir um domínio operativo da
arquitetura e do urbanismo da
\href{https://pt.wikipedia.org/wiki/Idade_Moderna}{Idade Moderna}
(séculos XV a XVIII) enquanto objeto de pesquisa historiográfica e
enquanto tradição fundadora do campo profissional, estético e técnico da
contemporaneidade, segundo os três aspectos seguintes:

\hypertarget{i-conhecer-os-fundamentos}{%
\paragraph{I Conhecer os fundamentos}\label{i-conhecer-os-fundamentos}}

Desenvolver uma visão de conjunto sobre os fundamentos urbanísticos e
construtivos sobre os quais se desenvolvem as culturas arquitetônicas da
era moderna.

\hypertarget{ii-saber-projetar}{%
\paragraph{II Saber projetar}\label{ii-saber-projetar}}

Dominar e ser capaz de empregar os elementos do vocabulário e a
gramática das tradições arquitetônicas da era moderna, com ênfase na
linguagem clássica.

\hypertarget{iii-saber-pesquisar}{%
\paragraph{III Saber pesquisar}\label{iii-saber-pesquisar}}

Refletir criticamente sobre a tradição historiográfica da arquitetura, e
especialmente sobre os debates a respeito das noções de Antiguidade e
Modernidade.

\hypertarget{metodologia}{%
\section{Metodologia}\label{metodologia}}

\hypertarget{considerauxe7uxf5es-gerais}{%
\subsection{Considerações gerais}\label{considerauxe7uxf5es-gerais}}

Neste semestre, TAU 0006 se desenvolve em modo \emph{remoto}. Todo o
conteúdo e as atividades da disciplina estão no ambiente virtual de
aprendizagem
\href{https://aprender3.unb.br/course/view.php?id=2766}{Aprender 3}. O
nosso fio condutor é o livro-texto; as videoaulas gravadas, as leituras
específicas e outros recursos indicados completam o conteúdo da
disciplina. Estaremos sempre em contato por videoconferência, mensagens
e fóruns, para discutir o conteúdo e as atividades ao longo do semestre.

Na área geral do Aprender 3, você vai encontrar:

\begin{itemize}
\tightlist
\item
  Este Plano de curso e o Cronograma da disciplina;
\item
  Mural de avisos postados pela equipe da disciplina, onde comunicamos
  informações gerais e eventuais alterações no cronograma ou nas
  atividades;
\item
  Fórum livre para discussões gerais relacionadas (ou não) à disciplina
  como um todo;
\item
  Link para acessar a videoconferência semanal ao vivo.
\end{itemize}

\hypertarget{desenvolvimento-do-semestre}{%
\subsection{Desenvolvimento do
semestre}\label{desenvolvimento-do-semestre}}

O semestre letivo começa com duas (ou mais) semanas de aproximação e
familiarização com o ambiente virtual de aprendizagem e com os demais
recursos eletrônicos da Universidade. Durante esse período inicial,
vamos esclarecer a abordagem da disciplina e nos familiarizar com a
organização do livro-texto, os recursos de interação ao vivo e o uso de
mensagens de texto nos fóruns de discussão do ambiente virtual.

O conteúdo propriamente dito da disciplina deve ser cumprido ao longo de
15 semanas. A matéria se divide em três Unidades com cinco semanas de
duração cada uma, e cada semana corresponde a um tópico de conteúdo.
Cada Unidade privilegia um dos três
\href{plano.md\#objetivos-de-aprendizagem}{Objetivos de aprendizagem} da
disciplina --- leia atentamente a exposição dos objetivos: são os
conhecimentos e as competências que esperamos de vocês ao final do
semestre. As Unidades estão organizadas, aproximadamente, em ordem
cronológica. Em cada Unidade, vamos desenvolver um trabalho
teórico-prático, pautado pelo Objetivo de aprendizagem que corresponde
àquela Unidade.

No ambiente virtual de aprendizagem
\href{https://aprender3.unb.br/course/view.php?id=2766}{Aprender 3} você
vai encontrar, para cada Unidade:

\begin{itemize}
\tightlist
\item
  Sumário do conteúdo;
\item
  Descrição do trabalho da Unidade e módulo de entrega;
\item
  Fóruns para discussão do conteúdo e do trabalho.
\end{itemize}

\hypertarget{desenvolvimento-das-aulas}{%
\subsection{Desenvolvimento das aulas}\label{desenvolvimento-das-aulas}}

Cada Tópico de conteúdo da disciplina foi previsto para ser cumprido em
uma semana, completando 15 semanas no semestre (sem contar as semanas de
adaptação inicial). Vamos acompanhar o conteúdo com {[}Resenhas{]}{[}{]}
e discussões semanais sobre a leitura do livro-texto e de outros textos
relevantes. Todas as leituras estão indicadas no Cronograma.

No ambiente virtual de aprendizagem
\href{https://aprender3.unb.br/course/view.php?id=2766}{Aprender 3} você
vai encontrar, para cada Tópico:

\begin{itemize}
\tightlist
\item
  Sumário do conteúdo;
\item
  Atividade de leitura semanal;
\item
  Videoaulas desenvolvendo alguns aspectos do conteúdo especialmente
  importantes para a sequência do semestre;
\item
  Notas de aula escritas (o conteúdo das videoaulas é idêntico ao das
  notas escritas).
\end{itemize}

Toda quinta-feira às 20h50 (horário de Brasília) teremos uma
videoconferência ao vivo na plataforma
\href{https://teams.microsoft.com}{Microsoft Teams} da UnB. Essa é uma
ocasião para tirar dúvidas sobre o conteúdo, as leituras e os trabalhos
em andamento.

\hypertarget{frequuxeancia-uxe0s-aulas}{%
\paragraph{Frequência às aulas}\label{frequuxeancia-uxe0s-aulas}}

Participar da videoconferência \emph{não é obrigatório}. No entanto, a
UnB considera oficialmente esta disciplina como sendo ``presencial'',
portanto temos que controlar a ``frequência'' ao longo do semestre.
Assim, você deve se fazer ``presente'' em pelo menos 75~\% das semanas
do semestre (12 de 15), participando ativa e produtivamente das
discussões com a turma. Para tanto, manifeste-se \emph{com propriedade}
nos diversos fóruns disponíveis no ambiente virtual de aprendizagem ou
na própria videoconferência (apenas estar ``logado'' na reunião, sem
participar ativamente, não conta). Para maiores detalhes, veja a seção
\href{plano-apoio-avalia.md\#apoio-ao-aluno}{Apoio ao aluno} mais
abaixo.

\hypertarget{uma-observauxe7uxe3o-importante-sobre-o-cronograma}{%
\paragraph{Uma observação importante sobre o
cronograma}\label{uma-observauxe7uxe3o-importante-sobre-o-cronograma}}

Dadas as circunstâncias atuais, o cronograma pode sofrer alterações
seguindo eventuais decisões da FAU ou dos órgãos superiores da UnB.
\emph{Respeitando o prazo previsto para cada unidade e o cronograma de
leituras semanais,} você pode cumprir o conteúdo (leituras e videoaulas)
e fazer as atividades (participação nos fóruns e entrega dos trabalhos)
no tempo e horário que for mais conveniente.

\hypertarget{programa}{%
\section{Programa}\label{programa}}

\hypertarget{unidade-i.-redes-urbanas-na-fundauxe7uxe3o-do-mundo-moderno}{%
\subsection{Unidade I. Redes urbanas na fundação do mundo
moderno}\label{unidade-i.-redes-urbanas-na-fundauxe7uxe3o-do-mundo-moderno}}

\begin{enumerate}
\def\labelenumi{\arabic{enumi}.}
\item
  \textbf{Rotas comerciais e paradigmas urbanísticos --} Primeira
  globalização: cidades mercantes da rota da Seda e do oceano Índico --
  Diversos modos de urbanizar: \emph{Kampung} malaio -- \emph{Boma}
  banto -- \emph{Campo} veneziano.
\item
  \textbf{Renascimento e historiografia --} História da arquitetura como
  história da construção: Fletcher e Choisy -- História da arquitetura
  como história do espaço: Argan -- Cúpulas europeias e explorações da
  técnica: Brunelleschi e De L'Orme -- Cúpulas asiáticas e malhas
  espaciais: influências dos reinos mongóis e turcos.
\item
  \textbf{Cidades ideais e cidades possíveis --} Macrocosmo e
  microcosmo, a tratadística italiana e o ideal geométrico na China --
  Urbanização conveniente: vilas novas em Portugal e \emph{jōkamachi} no
  Japão -- Forma geométrica e fortificação de Filarete a Vauban --
  Praças reais: plaza Mayor, place Royale, Square.
\item
  \textbf{Arte clássica --} Arquitetura como arte liberal: Alberti --
  Três momentos na reconstrução do classicismo -- Primeiro: resgatar o
  vocabulário antigo -- Segundo: estabelecer a gramática espacial
  clássica: Sangallo -- Artistas do Quatrocentos: perspectiva e projeção
  ortográfica.
\item
  \textbf{Arquitetura universal --} Três momentos na reconstrução do
  classicismo -- Terceiro: manipulando os elementos de arquitetura como
  articuladores do espaço: Bramante e Rafael -- Classicismo como
  ordenamento global do espaço: Miguel Ângelo -- Da ordem geométrica às
  ``ordens clássicas'': Serlio e Vinhola -- Projetos para a basílica de
  São Pedro.
\end{enumerate}

\hypertarget{unidade-ii.-linguagens-cluxe1ssicas-da-arquitetura}{%
\subsection{Unidade II. Linguagens clássicas da
arquitetura}\label{unidade-ii.-linguagens-cluxe1ssicas-da-arquitetura}}

\begin{enumerate}
\def\labelenumi{\arabic{enumi}.}
\setcounter{enumi}{5}
\item
  \textbf{Hidráulica ornamental e funcional --} Jardim como microcosmo:
  a \emph{villa} do maneirismo italiano e o pavilhão chinês -- Jardins
  monumentais: \emph{parterres} franceses e \emph{chahar-bagh}
  indo-persa -- Infraestrutura hidráulica na América: \emph{chinampas}
  astecas; \emph{andenes}, \emph{waru waru} e \emph{cochas} incaicos --
  Aterramento e canalização de águas urbanas: problemas bioclimáticos
  globais.
\item
  \textbf{Clássico anticlássico --} Plano e recessão: espaço
  escultórico, ordem colossal e cenografia -- Projeto tipológico:
  adaptações do classicismo a condicionantes espaciais e culturais --
  Permutações projetuais: repertório vitruviano na obra de Palladio --
  Desenho, mecenato e o ``problema do modo''.
\item
  \textbf{Barroco e classicismo --} Da tipologia espacial ao projeto
  geométrico: Bernini, Borromini e Guarini -- Elementos de arquitetura
  ou elementos de decoração: dilemas franceses -- Bioclimatismo e
  arquitetura clássica: iluminação, ventilação, sombreamento -- o
  \emph{hôtel} francês: soluções funcionais da modernidade.
\item
  \textbf{Arte acadêmica --} Colunata do Louvre: ponte entre o
  Renascimento e o Neoclassicismo -- Desenho, colorido e a Querela dos
  Antigos e dos Modernos -- Academias de arte: redefinindo a formação e
  o ofício da arquitetura -- O debate francês sobre o gosto.
\item
  \textbf{Urbanismo como arquitetura --} Traçados viários enquanto
  projetos arquitetônicos -- Tensão espacial barroca: reformas urbanas
  em Roma -- Desenhos desmedidos: Versalhes e seus êmulos -- Sincretismo
  decorativo: Pérsia safávida, império Otomano, Etiópia e Rússia.
\end{enumerate}

\hypertarget{unidade-iii.-estilo-e-territuxf3rio-inquietauxe7uxf5es-da-modernidade}{%
\subsection{Unidade III. Estilo e território: inquietações da
modernidade}\label{unidade-iii.-estilo-e-territuxf3rio-inquietauxe7uxf5es-da-modernidade}}

\begin{enumerate}
\def\labelenumi{\arabic{enumi}.}
\setcounter{enumi}{10}
\item
  \textbf{Barroco internacional --} Teorias do barroco: d'Ors, Tapié,
  Hautecœur -- Virtuosismo construtivo: estereotomia francesa, Sinan,
  Guarini -- De Nikkō à reconstrução do Tōdaiji: ápice da carpintaria
  japonesa -- Presságios: releituras do clássico em Wren, Vittone,
  Bertotti-Scamozzi e Juvarra.
\item
  \textbf{Universo em expansão --} Fronteiras da arquitetura no
  nascimento da história da arte: Fischer von Erlach, Winckelmann,
  Chambers -- Ruínas e crise da unidade clássica: \emph{Grand Tour} e a
  corrida das medições -- Pitoresco: refinamento cultural entre
  cosmopolitismo e nacionalismo -- Decoro arquitetônico e gêneros
  pictóricos: derivações tipológicas.
\item
  \textbf{Reação clássica --} Tradição, correção e inovação como
  problemas da crítica -- \emph{Kokugaku} e \emph{Yamato-e}: debates
  sobre caráter nacional no Japão -- Maturidade das tradições: da China
  manchu ao Mediterrâneo -- Classicismo para o povo: manuais de
  carpintaria e estampas populares no norte da Europa e na América
  anglófona
\item
  \textbf{Territórios e viação --} Território como escala de
  intervenção: reorganização espacial no mundo ibérico -- Arquitetura da
  cidadania: projetos urbanos na América do Norte -- Viação como
  estratégia de poder: École des Ponts et Chaussées e Tōkaidō --
  Topografia, agrimensura e representação: do \emph{Methodo lusitanico}
  de Serrão Pimentel à Malha continental de Jefferson.
\item
  \textbf{Primeiros modernos --} Teoria neoclássica e elogio da
  simplicidade formal: Lodoli, Milizia e Laugier -- Escola Politécnica:
  declínio da estereotomia e criação do cálculo estrutural -- Arquitetos
  da Revolução na França, Piranesi e a modernidade retrospectiva --
  Racionalismo estrutural: \emph{stick framing} e ferro.
\end{enumerate}

\hypertarget{apoio-ao-aluno}{%
\section{Apoio ao aluno}\label{apoio-ao-aluno}}

O conteúdo e as atividades da disciplina estão disponíveis no ambiente
virtual de aprendizagem
\href{https://aprender3.unb.br/course/view.php?id=2766}{Aprender 3}. A
equipe de TAU 0006 --- o professor, o estagiário docente e as monitoras
--- estará disponível ao vivo, por videoconferência no sistema
\href{https://teams.microsoft.com}{Microsoft Teams} da UnB, às
quintas-feiras no horário da aula. Além disso, cada Unidade tem um fórum
específico no Aprender 3 para tirar dúvidas sobre o desenvolvimento do
respectivo trabalho.

A disciplina pressupõe familiaridade com os recursos de pesquisa
bibliográfica \sout{presencial e} eletrônica da
\href{https://bce.unb.br}{Biblioteca Central (BCE) da UnB}, incluindo o
catálogo integrado e as diversas bases de dados de periódicos. A BCE
oferece tutoriais eletrônicos e cursos de capacitação frequentes para
familiarização com as suas ferramentas.

Usamos \emph{apenas} os sistemas oficiais da UnB para nos comunicarmos:
Aprender 3, Microsoft Teams e e-mail institucional. Verifiquem se o seu
acesso às plataformas e ao e-mail \texttt{@aluno.unb.br} está em dia!

\hypertarget{frequuxeancia}{%
\paragraph{Frequência}\label{frequuxeancia}}

Apesar de estar sendo conduzida de modo remoto, esta disciplina ainda é,
formalmente, \emph{presencial}. A frequência exigida por lei é de 75~\%
da carga horária, ou 12 das 15 semanas de conteúdo. Para registrar
presença em qualquer semana, você deve participar ativamente e com
contribuições pertinentes em qualquer um dos ambientes de discussão da
disciplina: manifestar-se na videoconferência semanal, postar nos fóruns
gerais da disciplina ou da Unidade em andamento, comentar nas resenhas
de leitura de colegas. Procure ter objetividade nas postagens ou falas e
contribuir positivamente, com questões, perguntas ou respostas
estruturadas e de interesse geral. \textbf{Atenção:} você pode postar
mais de um comentário por semana, mas o que vamos contar é o número de
\emph{semanas} em que você comentou!

\hypertarget{avaliauxe7uxe3o}{%
\section{Avaliação}\label{avaliauxe7uxe3o}}

\hypertarget{organizauxe7uxe3o-e-normas-gerais}{%
\subsection{Organização e normas
gerais}\label{organizauxe7uxe3o-e-normas-gerais}}

A nota final da disciplina é computada a partir do acompanhamento da
leitura do livro-texto e de outros textos indicados, além de três
trabalhos que correspondem a cada uma das três
\href{plano-programa.md}{Unidades do Programa}.

\begin{longtable}[]{@{}rlr@{}}
\caption{Atividades avaliativas da disciplina}\tabularnewline
\toprule
Unidade & Atividade & Peso\tabularnewline
\midrule
\endfirsthead
\toprule
Unidade & Atividade & Peso\tabularnewline
\midrule
\endhead
& Resenha e discussão de textos & 10 \%\tabularnewline
I & Análise urbana & 30 \%\tabularnewline
II & Linguagens clássicas & 30 \%\tabularnewline
III & Pesquisa e crítica & 30 \%\tabularnewline
\bottomrule
\end{longtable}

A entrega de todas as atividades da disciplina será feita,
exclusivamente, por meio eletrônico, na plataforma
\href{https://aprender3.unb.br/course/view.php?id=2766}{Aprender 3} da
UnB, nos fóruns e tarefas previstos para esse fim. Não recebemos,
\emph{sob hipótese alguma}, trabalhos por e-mail. A participação na
videoconferência \emph{não substitui} a discussão dos textos por
escrito.

\hypertarget{pluxe1gio}{%
\paragraph{Plágio}\label{pluxe1gio}}

Integridade e reconhecimento são valores éticos fundamentais da
cidadania e da profissão. Apropriação do trabalho alheio --- seja de
colegas de turma, da bibliografia, ou de material ``garimpado'' na
Internet --- é uma prática inaceitável e resultará em atribuição da nota
0 (zero) ao trabalho que incorrer nela. Dependendo da gravidade do fato,
podemos relatar o ocorrido ao colegiado de graduação da FAU.

\hypertarget{crituxe9rios-de-avaliauxe7uxe3o}{%
\subsection{Critérios de
avaliação}\label{crituxe9rios-de-avaliauxe7uxe3o}}

Os critérios de avaliação de cada atividade são apresentados com a
respectiva ementa. Em linhas gerais, o que esperamos de todas as
entregas é atenção ao que foi pedido, pesquisa e leitura criteriosa do
material necessário à realização do trabalho, reflexão crítica, clareza
e capricho na execução. Para obter aprovação na disciplina, você precisa
atender a \emph{todos} os três requisitos seguintes:

\begin{itemize}
\tightlist
\item
  Pontuar em 10 das 15 resenhas (entregar a resenha e obter a nota 1);
\item
  Entregar os três trabalhos das Unidades.
\item
  Obter a média final de 5,0 (MM).
\item
  Participar ativamente das atividades propostas, para cumprir o
  requisito de frequência.
\end{itemize}

\hypertarget{resenha-e-discussuxe3o-de-texto}{%
\paragraph{Resenha e discussão de
texto}\label{resenha-e-discussuxe3o-de-texto}}

Demonstrar compreensão do texto e reflexão crítica sobre o conteúdo e o
modo de apresentação deste.

\hypertarget{trabalhos-das-unidades}{%
\paragraph{Trabalhos das Unidades}\label{trabalhos-das-unidades}}

Demonstrar capacidade de ir além do conteúdo mínimo das leituras,
pesquisando e aprofundando o entendimento sobre a arquitetura e o
urbanismo nas tradições da Idade Moderna e, principalmente, na linguagem
clássica da arquitetura. \textbf{Atenção:} embora o trabalho de cada
unidade esteja focado no respectivo
\href{plano.md\#objetivos-de-aprendizagem}{Objetivo de aprendizagem}, em
cada trabalho você deve demonstrar o conhecimento do conteúdo de modo
\emph{cumulativo} desde o início do semestre.

\hypertarget{um-esclarecimento-sobre-criatividade-e-originalidade}{%
\paragraph{Um esclarecimento sobre criatividade e
originalidade}\label{um-esclarecimento-sobre-criatividade-e-originalidade}}

Numa faculdade de Arquitetura, somos muitas vezes instados a inventar
uma ``expressão individual'' e a demonstrar ``originalidade''. Em se
tratando de arquitetura e urbanismo tradicionais e da pesquisa
histórica, contudo, a nossa exigência de originalidade é simples: não
copiar o trabalho alheio e não se apropriar das ideias de outrem sem dar
o devido crédito. Aspiramos à \emph{qualidade} antes que à
\emph{novidade}, e consideramos que a criatividade consiste em resolver
com apuro e propriedade um determinado problema de projeto ou pesquisa,
baseando-se nos repertórios das tradições arquitetônicas e nos métodos e
referências da pesquisa histórica.

\hypertarget{bibliografia}{%
\section{Bibliografia}\label{bibliografia}}

A bibliografia abaixo é uma indicação de aprofundamento, e não é
exaustiva. Vocês são responsáveis por pesquisar e consultar qualquer
outra bibliografia de apoio que se faça necessária à consecução das
avaliações desta disciplina. Para conveniência, dividimos a bibliografia
em quatro partes:

\begin{itemize}
\tightlist
\item
  \emph{\protect\hyperlink{leituras-obrigatuxf3rias}{Leituras
  obrigatórias}:} os textos que vamos estudar detalhadamente e resenhar
  ao longo do semestre;
\item
  \emph{\href{biblio-fontes.md}{Tratados e fontes primárias}:} Obras
  criadas por alguns dos principais arquitetos e escritores dos séculos
  XV a XVIII, são os textos fundadores da nossa disciplina e da visão
  moderna sobre a história e a teoria da arquitetura;
\item
  \emph{\href{biblio-basica.md}{Bibliografia básica}:} alguns textos
  canônicos sobre história da arquitetura publicados no século XX, que
  constam da ementa oficial da disciplina.
\item
  \emph{\href{biblio-complementar.md}{Bibliografia complementar}:} uma
  variedade de referências relevantes para a discussão dos tópicos do
  semestre, que forma a base da nossa abordagem didática.
\end{itemize}

\hypertarget{leituras-obrigatuxf3rias}{%
\subsection{Leituras obrigatórias}\label{leituras-obrigatuxf3rias}}

O livro-texto de base que nos acompanhará ao longo de todo o semestre
foi concebido pelos fundadores do {[}Global Architectural History
Teaching Collaborative (GAHTC){]}{[}6{]}, os historiadores da
arquitetura Mark Jarzombek e Vikramaditya Prakash, em parceria com o
famoso autor de manuais técnicos Francis D. K. Ching:

\begin{quote}
Ching, Francis D. K., Mark M. Jarzombek, e Vikramaditya Prakash.
\emph{História global da arquitetura.} Traduzido por Alexandre
Salvaterra. 1.ª ed.~São Paulo: Senac São Paulo, 2016.
\end{quote}

Aproveite para conhecer os recursos que o site do GAHTC oferece!

Além do livro-texto, também vamos ler um pequeno clássico do arquiteto
inglês John Summerson (1904--1992), que nos guiará com mais detalhe pelo
universo da arquitetura clássica europeia:

\begin{quote}
Summerson, John. \emph{A linguagem clássica da arquitetura.} Traduzido
por Sylvia Ficher. São Paulo: Martins Fontes, 1997.
\end{quote}

Perto do final do semestre, vamos ler um breve texto do historiador da
arquitetura espanhol Carlos Sambricio sobre as reformas territoriais do
século XVIII, que o autor apresentou em Brasília em 2014:

\begin{quote}
Sambricio, Carlos. ``Projetos espaciais na América espanhola no último
terço do século XVIII: economia, política e ordenação territorial''. In
\emph{Tempos e escalas da cidade e do urbanismo: quatro palestras,}
organizado por Elane Ribeiro Peixoto, Maria Fernanda Derntl, Pedro P.
Palazzo, e Ricardo Trevisan, traduzido por Pedro P. Palazzo e Elane
Ribeiro Peixoto, 29--59. Brasília: Faculdade de Arquitetura e Urbanismo
da Universidade de Brasília, 2014.
\end{quote}

Por fim, indicamos o tratado contemporâneo a seguir, elaborado pelo
arquiteto inglês Robert Chitham (1935--2017), como uma referência
completa e de fácil compreensão sobre as ordens clássicas:

\begin{quote}
Chitham, Robert. \emph{The Classical Orders of Architecture.} 2.ª ed.
Amsterdam: Architectural Press, 2005.
\end{quote}

\nocite{*}

\printbibliography[title={Tratados e fontes primárias},heading=subbibintoc,keyword={Fontes}]

\printbibliography[title={Bibliografia Básica},heading=subbibintoc,keyword={Básica}]

\printbibliography[title={Bibliografia Complementar},heading=subbibintoc,keyword={Complementar}]

\printbibliography

\end{document}
